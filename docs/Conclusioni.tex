\capitolo{Conclusioni}

Gli obiettivi che questo lavoro si poneva era essenzialmente due:

\begin{enumerate}
\item costruire una specializzazione della classe astratta MCFLSSolver che implementasse le metodologie introdotte da O. E. Livne e A. Brandt, di cui ne esiste una implemntazione in Matlab~\cite{lamg_code}, in modo da poterle utilizzare all'interno dei codici per il Flusso di Costo Minimo sviluppati in Dipartimento;\\
\\
\item testarne la bontà e confrontarne le prestazioni con gli altri risolutori presenti nel framework.
\end{enumerate}

Per quanto riguarda il primo obiettivo questo può ritenersi raggiunto in quanto il codice compila senza errori all'interno del framework.

Il secondo invece può ritenersi solo parzialmente raggiunto poiché è stato possibile ottenere solo dei risultati parziali che non sono stati quindi riportati.

È significativo, per concludere, spendere qualche parola sugli strumenti esterni utilizzati, ossia Matlab e la libreria Blaze.
\\
Matlab è un linguaggio molto comodo quando si deve lavorare con le matrici.
Nonostante la sintassi delle librerie Blaze sia semplice ed intuitiva, questa non può reggere il confronto con la semplicità con la quale vengono manipolate le matrici nell'ambiente Matlab.
Ad esempio se \inlinecode{X} è un vettore con \inlinecode{n} componenti in Matlab è possibile invertire l'ordine di questi elementi con il comando
\begin{codice}
X(n:-1:1)
\end{codice}

È anche possibile utilizzare due vettori come subscripts per individuare una sottomatrice. 
Se ad esempio \inlinecode{V} è un vettore ad \inlinecode{m} componenti e \inlinecode{W} uno ad \inlinecode{n} componenti è possibile individuare una sottomatrice di \inlinecode{A} utilizzando il comando
\begin{codice}
A(V,W)
\end{codice}
oppure
\begin{codice}
A(:,V) 
\end{codice}
dove l'operatore duepunti indica a Matlab di selezionare tutte le righe della matrice \inlinecode{A} per costruire la sottomatrice.

O ancora è possibile scrivere
\begin{codice}
degree = sum(A ~= 0, 1)
\end{codice}
ed ottenere come risultato un vettore degree che contiene un numero di elementi pari alle righe della matrice \inlinecode{A}, ciascuno dei quali assume un valore pari al numero degli elementi nonzero della riga corrispondente.

Questo, unito ad un uso eccessivo di factory all'interno del codice essenzialmente dovuto a necessità delle versioni precedenti di testare la bontà di differenti metodologie, ha reso il lavoro di comprensione del codice particolarmente articolato.

La libreria Blaze comunque è in continuo sviluppo. 
L'attuale versione fornisce supporto per la creazione di viste di tipo sottomatrice mediante l'utilizzo degli Smart Expression Templates, con comportamenti simili all'operatore parentesi di Matlab.

Tale operazione, che attualmente è stata implementata manualmente nella classe MtxOps, è molto utile per eseguire in maniera efficiente le operazioni di Elimination e Aggregation

Si procederà dunque ad investigare sull'efficacia del suo utilizzo in una delle prossime revisioni del codice.