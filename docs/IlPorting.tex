\capitolo{Il Porting}


Per il porting è stato utilizzato il linguaggio C++ poiché il framework MCFUniPi in cui si va ad inserire è interamente scritto in questo linguaggio.
La scelta della libreria matematica da utilizzare è ricarduta su Blaze-lib\footnote{\url{https://code.google.com/p/blaze-lib/}}.\\
\\
Blaze è una libreria matematica opensource ad alte performance, scritta in C++, che consente di effettuare calcoli con matrici sparse e dense.

Utilizza il paradigma degli \emph{Smart Expression Templates}per combinare l'eleganza del codice e la facilità di utilizzo con performance elevate, rendendola una delle più intuitive e veloci librerie matematiche in C++ disponibili~\cite{blaze_SET}.\\
\\
Il seguente esempio mostra come l'implementazione tramite le libreria Blaze del metodo del gradiente coniugato (CG), definita nell'algoritmo~\vref{alg:CG}, sia molto semplice e molto simile alla formulazione matematica.
\begin{codice}[]
const size_t NN( N*N );

blaze::CompressedMatrix<double,rowMajor> A( NN, NN );
blaze::DynamicVector<double,columnVector> x( NN, 1.0 );
blaze::DynamicVector<double,columnVector> b( NN, 0.0 );
blaze::DynamicVector<double,columnVector> r( NN );
blaze::DynamicVector<double,columnVector> p( NN );
blaze::DynamicVector<double,columnVector> Ap( NN );
double alpha, beta, delta;

// ... Inizializzazione della matrice sparsa A

// Algoritmo del Gradiente Coniugato
r = b - A * x;
p = r;
delta = (r,r);

for( size_t iteration=0UL; iteration<iterations; ++iteration )
{
   Ap = A * p;
   alpha = delta / (p,Ap);
   x += alpha * p;
   r -= alpha * Ap;
   beta = (r,r);
   if( std::sqrt( beta ) < 1E-8 ) break;
   p = r + ( beta / delta ) * p;  
   delta = beta;
}
\end{codice}

È evidente come la parte principale del codice, il loop for, sia molto simile alla formulazione matematica dell'algoritmo rendendo il codice facile da leggere e più manutenibile. Inoltre, proprio per l'utilizzo degli SET, le performance del codice sono molto vicine al picco teorico\footnote{\url{https://code.google.com/p/blaze-lib/wiki/Benchmarks}}.


\begin{algorithm}
\caption{$x = CG(x_0,A,b)$}\label{alg:CG}
\begin{algorithmic}[1]
\State $r_0 = b -Ax_0$
\State $p_0 = r_0$
\State $k=0$
\Loop
	\State $\alpha_k = \frac{r_k^Tr_k}{p_k^TAp_k}$
	\State $x_{k+1} = x_k + \alpha_kp_k$
	\State $r_{k+1} = r_k - \alpha_kAp_k$
	\If ($r_{k+1}$ is sufficiently small)
		\State exit loop
	\EndIf
	\State $\beta_k = \frac{r_{k+1}^Tr_{k+1} }{r_k^Tr_k }$
	\State $p_{k+1} = r_{k+1} + \beta p_k$
	\State $k = k+1$
\EndLoop
\end{algorithmic}
\end{algorithm}


Il codice di LAMG è scritto in Matlab.
\\
Matlab (MATrix LABoratory) è un ambiente di calcolo numerico e un linguaggio di programmazione sviluppato da MathWorks. 
Matlab consente di manipolare matrici, disegnare grafici e funzioni, implementare algoritmi, creare interfacce utente ed è possibile interfacciarlo con programmi scritti in altri linguaggi, compresi C, C++, Java e Fortran.\\
\\
L'applicazione Matlab è costruita intorno al linguaggio Matlab, ed il suo utilizzo principale consiste nell'inserire codice Matlab nella finestra dei comandi, che funge da shell matematica interattiva, o nell'eseguire file testuali che contengono codice Matlab, inclusi script e funzioni.\\
\\
È possibile definire variabili utilizzando l'operatore di assegnamento.
Il linguaggio Matlab è un linguaggio \emph{dinamico} ed è possibile assegnare valori alle variabili senza la necessità di dichiararne il tipo, a meno che non debbano essere trattate come oggetti simbolici, e questo tipo può cambiare.\\
Matlab supporta inoltre la programmazione ad oggetti e permette di definire classi, derivazione, packages e dispatch virtuale dei metodi.\\
\\
LAMG implementa le funzioni più time-consuming in C++, utilizzando i Matlab Executables (\mex files) per interfacciare il codice.

Quando si scrivono programmi in C/C++, si assume sempre che il programma inizi l'esecuzione invocando la funzione \inlinecode{main()}. I \mex files hanno un comportamento simile, poiché iniziano sempre invocando una funzione speciale chiamata mexFunction.
Questa funzione, la cui firma è riportata in tabella~\vref{code:mexFunction}, svolge il ruolo di gateway tra la chiamata in Matlab e l'effettiva implementazione in C.

\begin{table}
\caption{La firma della mexFunction}
\label{code:mexFunction}
\begin{codice}
//Si possono usare tutti gli header
#include "math.h"
#include "mex.h"   //Si deve includere mex.h

void mexFunction(int nlhs, mxArray *plhs[],\
					int nrhs, const mxArray *prhs[])
{
    //Codice C/C++ della funzione
	...   
    return;
}
\end{codice}
\end{table}

Per creare una mexFunction è necessario includere la libreria \inlinecode{mex.h} che contiene tutte le API fornite da Matlab.
Questa funzione ha quattro parametri che corrispondono al modo in cui si crea una funzione in Matlab:
\begin{itemize}
\item \inlinecode{nlhs} è un intero che ci dice il numero di argomenti ``left hand side'', cioè quanti sono i valori che vengono ritornati dalla funzione.
\item \inlinecode{plhs} è un puntatore ad un array di mxArray che contiene gli argomenti che verranno effettivamente tornati a
Matlab.
\item \inlinecode{nrhs} è un intero che indica il numero di parametri della funzione.
\item \inlinecode{prhs} è un puntatore costante agli argomenti della funzione.
\end{itemize}

Supponiamo di avere la funzione Matlab \inlinecode{function seq = sequence(X,symA,symB)} e di volerla riscrivere in C.
Tale funzione prende in input un vettore \inlinecode{X} di lunghezza n e due simboli \inlinecode{symA} e \inlinecode{symB} e ritorna una sequenza di n caratteri consecutivi in cui l'i-esimo simbolo è \inlinecode{symA}, se il corrispondente valore di \inlinecode{X} è pari, oppure \inlinecode{symB} se questo è dispari.
La corrispondente mexFunction avrà questa forma:

\begin{codice}
#include "mex.h"
void mexFunction(int nlhs, mxArray *plhs[],\
					int nrhs, const mxArray *prhs[]){
	mwSize i,n;

	n = mxGetN(X_IN); //Dimensione dell'array X
	plhs[0] = mxCreate(n,mxChar);
	for(i=0; i<n; i++)
		plhs[0][i] = (prhs[0][i] % 2 == 0 ) ? prhs[1] : prhs[2];
\end{codice}

Il codice di questa funzione dovrà poi essere salvato nel file \inlinecode{sequence.c} a cui dovremo anche affiancare un file dal nome \inlinecode{sequence.m}. In questo modo, una volta compilato il file C, quando si invocherà la funzione \inlinecode{sequence} da Matlab, verrà eseguito il codice della mexFunction.
In questo esempio \inlinecode{nlhs == 1} e \inlinecode{plhs[0]} punta ad un tipo di dato allocato con la chiamata a \inlinecode{mxMalloc()}. 
Una volta terminata la computazione sarà possibile accedere a quel tipo di dato dall'ambiente Matlab poiché verrà tornato dalla funzione invocata. Inoltre \inlinecode{nrhs == 3} ed i tre indici di \inlinecode{prhs} puntano ad \inlinecode{X}, \inlinecode{symA} e \inlinecode{symB} rispettivamente. In questo modo le strutture dati create nell'ambiente di Matlab sono visibili anche nel codice C.
È anche possibile accedere direttamente dal codice C a tutte le funzionalità e a tutte le strutture dati che Matlab rende disponibili attraverso le sue API, tramite l'inclusione del file \inlinecode{``mex.h''}.

\sezione{La struttura di LAMG}

\fig{Grafica/PackageLAMG}{Struttura dei package di LAMG}{fig:LAMG}

Il codice di LAMG ha un'architettura complessa, come si può vedere in figura~\vref{fig:LAMG}. Questo perché oltre che poter essere utilizzato come risolutore di sistemi lineari della forma
\begin{equation*}
Ax = b
\end{equation*}
utilizzando la tecnica del multigrid algebrico è possibile impiegarlo per risolvere il problema di trovare il più piccolo autovalore diverso da $0$ utilizzando l'Exact Interpolation Scheme (EIS)~\cite[§5.4]{lamg_Report}, può essere impiegato come precondizionatore per metodi del PCG.

I sorgenti inoltre contengono classi che rappresentano grafi e ne consentono una visualizzazione a video, classi che consentono di interfacciare LAMG con la University of Florida Sparse Matrix Collection\footnote{\url{http://www.cise.ufl.edu/research/sparse/matrices}}.
Infine nei sorgenti è disponibile una classe che consente di eseguire test contro altri risolutori che impiegano tecniche analoghe.

Il nostro lavoro si è concentrato sul porting del package \inlinecode{amg}, che contiene l'implementazione di LAMG. Questo è suddiviso nei seguenti subpackage:

\begin{itemize}
\item \inlinecode{amg.api} che contiene la definizione delle interfacce Builder, HasOptions, HasArgs e IterativeMethod oltre alla definizione della classe Options che contiene tutte le opzioni di default di LAMG.
\item \inlinecode{amg.coarse} che contiene le definizioni delle interfacce Aggregator, CoarseSet e AggregationStrategy e delle relative implementazioni, che vengono utilizzate durante la fase di aggregation.
\item \inlinecode{amg.level} che contiene l'implementazione delle strutture dati e delle operazioni utilizzate nella fase di setup per generare i livelli del ciclo multigrid.
\item \inlinecode{amg.setup} contiene le classi relative alla fase di setup, cioè quelle che consentono di effettuare il coarsening per generare la gerarchia multilivello.
\item \inlinecode{amg.tv} questo package contiene la definizione di InitialGuess per i test vector, diverse politiche di guessing ed una factory.
\item \inlinecode{amg.relax} contiene le politiche di rilassamento che vengono utilizzate durante la fase di solve, nella risoluzione del sistema ridotto.
\item \inlinecode{amg.solve} contiene il SolverLamg, un decorator per SolverLamgLaplacian. Questa classe rende di fatto possibile la risoluzione di problemi in cui la matrice dei coefficienti è almeno simmetrica e a predominanza diagonale (SDD), aumentando il sistema ad una laplaciana.
\end{itemize}

I sorgenti di LAMG infine, sono corredati da esempi tipici di utilizzo. Di seguito si riporta quello che meglio corrisponde alle necessità di MCFUniPi, in cui cioè la matrice dei coefficienti è una laplaciana.

\begin{codice}[commandchars=\\\{\}]
g = Graphs.grid('fd', [40 40], 'normalized', true);
A = g.laplacian;
%Fase di setup \label{code:beginSetup}
inputType = 'laplacian';
solver = 'lamg';
lamg    = Solvers.newSolver(solver, 'randomSeed', 1);\label{code:factory}
setup   = lamg.setup(inputType, A);\label{code:setup}

%Fase di solve
b = rand(size(A,1), 1);
b = b - mean(b);
[x, ~, ~, details] = lamg.solve(setup, b, 'errorReductionTol', 1e-8);
\end{codice}

\sezione{Architettura di LAMGLSSolver}

Il solver implementato ha la struttura riportata in figura~\vref{fig:classiLAMG}

\fig{Grafica/ClassiLAMG}{Diagramma UML delle classi di LAMGLSSolver}{fig:classiLAMG}

La classe \inlinecode{MtxOps} contiene tutte le operazioni sulle matrici che non sono, o non erano, implementate direttamente dalla libreria Blaze, come ad esempio l'estrazione di sottomatrici di una matrice data, di sottovettori e la norma-2 di un vettore.
\\
La classe \inlinecode{Encapsulator} ha lo scopo di contenere una matrice di un tipo arbitrario, scelto tra sparsa e densa. 
Tale necessità insorge dal momento che la matrice dei coefficienti $A^l$, che per il livello $l=1$ è sparsa, tende a densificarsi durante il processo di coarsening e non è possibile sapere a priori quale sarà il livello $l$ in cui sarà necessario, per motivi di prestazioni, passare da una matrice sparsa ad una densa.\\
A supporto di tale decisione viene utilizzato il metodo \inlinecode{isSparseOK()} che controlla, ad ogni costruzione di livello, se è il momento di passare all'utilizzo delle matrici dense.\\
Il criterio utilizzato tiene conto del numero dei nonzero della matrice. Quando questi superano o si avvicinano al $50\%$ del numero totale degli elementi conviene passare all'utilizzo delle matrici dense\footnote{K. Iglberger, conversazione privata.}.

\sottosezione{LAMGLSSolver}

L'implementazione della classe \inlinecode{LAMGLSSolver} ha rappresentato la parte più consistente del lavoro.
Questa classe è una specializzazione diretta della classe \inlinecode{MCFLSSolver} discussa nel capitolo~\vref{cap:Framework} ed implementa un risolutore di sistemi lineare nella forma
\begin{equation*}
Ax = b
\end{equation*}
utilizzando le tecniche del multigrid algebrico visto nel capitolo~\vref{cap:LAMG}.
In particolare
\begin{itemize}
\item ridefinisce quei metodi la cui implementazione base, nella classe padre, non è sufficiente;
\item implementa il metodo \inlinecode{SolveADAT()} presente in \inlinecode{MCFLSSolver};
\item implementa le strategie di coarsening di LowDegreeElimination ed Aggregation.
\end{itemize}

Di seguito vengono descritti i metodi e le strutture dati implementati, soffermandosi sugli aspetti più importanti ed evidenziandone le caratteristiche principali.

\subsubsection{bool SetD()}
Questo metodo viene invocato dopo il metodo \inlinecode{SetGraph()}, nel quale si ricevono le informazioni circa la topologia del grafo, e riceve come input il peso di ciascun arco.
Con queste informazioni è possibile costruire la matrice Laplaciana del primo livello e, a partire da questa, generare l'intera gerarchia del multigrid.

Tale compito viene delagato al metodo privato \inlinecode{buildMultigrid()} il cui comportamento è riassunto in figura~\vref{figurinaBellina}.

\fig{Grafica/diagrammaCreazioneMultigrid}{Diagramma di flusso della fase di setup}{figurinaBellina}

Allo stato Eliminate corrisponde una chiamata al metodo privato \inlinecode{coarsenElimination()} mentre a quello Aggregate il metodo \inlinecode{coarsenAggregation()}.

La procedura si interrompe se dopo la creazione di un qualsiasi livello viene tornato \inlinecode{true} dal metodo \inlinecode{isrelaxationFast()}.

\subsubsection{Level* coarsenElimination()}

Questo metodo implementa la strategia di coarsening Eliminate che corrisponde all'algoritmo~\vref{alg:Elimination}.
Provvede quindi ad individuare gli eventuali nodi lowdegree e ad eliminarli, costruendo la nuova matrice dei coefficienti per mezzo del Galerkin coarsening.
Se per questa matrice è stato possibile eliminare nodi allora viene restituito un nuovo livello, contenente la matrice appena calcolata e una copia della matrice di proiezione così costruita, e questo viene aggiunto alla gerarchia multigrid.\\
Altrimenti viene tornato \inlinecode{NULL}. In ogni caso si procede ad aggregare i nodi rimasti. 

\subsubsection{Level* coarsenAggregation()}

uesto metodo implementa la strategia di coarsening Aggregate che corrisponde all'algoritmo~\vref{alg:Aggregate}.

Provvede quindi a calcolare le affinity tra i vari nodi del grafo e ad effettuare $n=2$ stage di Aggregazione selezionando poi l'insieme aggregato migliore.
Se al termine di questa procedura non è stato possibile aggregare dei nodi o se la migliore aggregazione individuata non è stata soddisfacente, perché ad esempio riduceva il numero di nodi di un fattore troppo piccolo, il metodo ritorna \inlinecode{NULL} ed il coarsening termina.
Altrimenti restituisce un nuovo livello che viene aggiunto al multigrid. Si prosegue quindi nella costruzione del multigrid tentanto di eliminare di nuovo eventuali nodi lowdegree presenti nel nuovo grafo.

Questi metodi sono stati implementati come metodi template poiché, come detto, non è possibile conoscere a priori se la matrice dei coefficienti da analizzare sarà di tipo denso o sparso.
É stato comunque implementata una versione specializzata per ciascuno di questi tipi di matrici, al fine di ottimizzare alcune procedure.

\subsubsection{Gerarchia dei Livelli}

Per implementare i livelli del multigrid si è deciso di utilizzare la gerarchia di classi riportata in figura~\vref{label2}.

\fig{Grafica/gerarchiaClassiLivello}{Diagramma UML}{label2}

La classe \inlinecode{Level} fornisce una interfaccia ed una implementazione di default dei metodi comuni a tutti i livelli.
I metodi \inlinecode{restrict()}, \inlinecode{interpolate()}, \inlinecode{coarseType} e \inlinecode{relax()} vengono utilizzati durante la fase di solve e corrispondono alle omonime operazioni.
I metodi \inlinecode{canCoarsen()} e \inlinecode{isRelaxationFast()} determinano, durante la fase di setup, se è possibile e conveniente proseguire con la costruzione del multigrid.

Ciascuna specializzazione sovrascrive i metodi per i quali è necessario ridefinire il comportamento. Ognuna di queste sottoclassi corrisponde ad un livello creato con l'omonima strategia di coarsening, mentre il livello \inlinecode{LevelFinest} corrisponde al sistema di partenza.

\subsubsection{LSSStatus SolveADAT()}

Questa procedura utilizza il metodo privato \inlinecode{solveCycle()} per eseguire uno o più cicli multigrid al fine di individuare la soluzione.\\
In particolare
\begin{itemize}
\item applica un'iterazione multigrid utilizzando i metodi forniti dalla gerarchia di classi \inlinecode{Level}.
\item controlla se l'approssimazione corrente della soluzione ha raggiunto la precisione desiderata;
\item se ciò accade, e se non ha esaurito la risorsa tempo a lui assegnata, effettua un altro ciclo.
\end{itemize}

La precisione desiderata viene raggiunta se e quando viene verificata la seguente condizione:
\begin{equation}
\label{eqn:Prcsn}
|b - A\bar{x}_n | \leq Prcsn
\end{equation}

dove $Prcsn$ è un vettore che viene passato al risolutore invocando il metodo \inlinecode{SetPrcsn()}.

Se la condizione~\eqref{eqn:Prcsn} è verificata il metodo deve ritornare \inlinecode{kOk} e copiare la soluzione nel parametro \inlinecode{Res}.
Altrimenti il metodo può riapplicare una nuova iterazione di multigrid usando come approssimazione iniziale l'approssimazione appena individuata.\\
\\
Può accadere che il solver non possa eseguire altre iterazoni poiché ha esaurito la risorsa tempo che gli è stata assegnata: in tal caso deve tornare \inlinecode{kLwPrcsn}.
Il chiamante, nel nostro caso l'algoritmo IP, ha la possibilità di invocare di nuovo il metodo \inlinecode{SolveADAT()} se ritiene di necessitare di una soluzione migliore, concedendo di fatto al solver un'ulteriore iterazione.
Se dopo questa chiamata non è stato possibile raggiungere la precisione desiderata si deve tornare \inlinecode{kNError}.
